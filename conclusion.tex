To conclude, both programs are very faulty. Whilst the format string vulnerability doesn't cause major issues in
isolation, the buffer overflow attack allows any user to gain root access to the system, which is a very serious
problem. A format string vulnerability can, however, be used to nullify certain countermeasures against buffer
overflows. The security flaws exposed here are caused by problems which good programming practice can completely
prevent, but which still exist in real-world scenarios to this day. Though few new systems will be subject to these
flaws, the cost of rebuilding a pre-existing system which already exhibits these flaws can be prohibitively high. For
this reason, new programs should be built with security as a priority and throughly checked prior to deployment so a
user needn't pay the excessive costs associated with a security breach.
