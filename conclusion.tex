In conclusion, we believe there are 3 major points to take away: 

\begin{itemize} 

\item Binary exploits such as format string vulnerabilities and buffer overflows are not unavoidable by-products of
computing. They can be prevented with sufficiently aware design. Educating programmers is the best approach, but for
retroactive patching there are automated tools to make detection (and in many cases repair) of these vulnerabilities
possible.

\item Left unattended, these exploits can cause significant damage: even if they are prevented from hijacking a process
by measures such as NX regions, they can still cause failures. In the real world, anything that causes mission-critical
software to deviate from intended behaviour in any way can be disastrous.

\item The preventative and curative measures we have discussed make problems immediately obvious to the developer when
low-level languages are used, but as more and more software is developed in higher languages, and with larger libraries,
it is becoming more important than ever to ensure new code is safe. Automatic measures can mitigate a lot of damage,
but there is no substitute for security-aware programming.

\end{itemize}

