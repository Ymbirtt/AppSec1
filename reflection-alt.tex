\subsection{Causes and Preventative Measures}
Both attacks are entirely preventable, with remedies being extremely simple and indeed automatic in new cases.
In the case of format string vulnerabilities, the entire issue can be avoided by not using user-provided content in
any part of the format string. If this is required for some unknown reason, then careful parsing of the string 
is needed beforehand, to guard against improper specifier usage. Tools such as {\tt Splint} can detect 
this problem in a codebase\cite{splint_art}, and requiring programmers to pass static analysis will prevent most,
if not all, occurences. We believe that the main reason this class of vulnerability exists is poor or lazy
coding, and cracking down on that is the best method of prevention. The complexity of patching this problem can vary,
but the detection is straightforward and automatic.

Similarly, in new code, buffer overflow situations can be easily avoided. They occur only where functions that ignore
buffer sizes are used, and for almost all of these functions there are safe alternatives. Again, linter tools can
detect many instances of this vulnerability, meaning that existing codebases in low-level languages can be patched.
Preventing code from offering opportunities for buffer overflows at the compilation stage is now quite easy; 
Stack protectors (also called canaries) are now compiled in automatically by default, and provide a very effective
method of detecting stack smashing at the point of a function returning. If a smash is detected, then the program crashes.
Later versions of the C standard library and compilers also include a macro definition called {\tt FORTIFY\_SOURCE},
which will attempt to automagically replace unsafe functions (such as {\tt strcpy}) with safer variants, such as 
(such as {\tt strncpy})\cite{fort_source}. Microsoft also provide a header file, {\tt banned.h}, which will cause
usage of unsafe functions to throw compile time errors\cite{banned}.

We believe that lazy design is still a problem, but modern compilation toolchains provide excellent safeguards against
casual mistakes which would have previously been highly dangerous. Further enhancements include ASLR (address space
layout randomisation) and NX (non-executable) regions; they make placing code that can be executed much harder,
and are enabled by default in modern operating systems and development toolchains\cite{wiki_aslr}\cite{wiki_nx}.
It is interesting to note are that ASLR was added slowly in small steps to various systems, meaning that it didn't
mitigate all attacks straight away. It is entirely software based, unlike NX regions, which require hardware support.
Furthermore it seems that whilst Ubuntu 9.11 upwards, with the i386 architecture, should have an emulation of NX regions
enabled by default\cite{nx_bit}, the SEED virtual machines seem to have disabled them. 
Presumably this was for the educational purposes of these labs.

The common NOP sled is quite detectable in principle, so emulated execution of a program can use simple heuristics to determine
if a basic sled is being deployed. Very few programs would have a legitimate reason for using multiple consecutive NOPS\cite{zip_quine},
and we believe that is why Drum's metasploit-generated shellcode used a method of generating a NOP sled that didn't actually
use NOP instructions - simply using regular instructions in a way that has no side-effects and does no practical work\cite{wiki_sled}.

All that aside, we believe the best defense is simply to educate programmers about the dangers of their software being attacked, and
how they can design against it.

\subsection{Damage Potential}

By itself, a format string vulnerability offers only limited opportunity for control flow execution, and typically
relies on altering memory locations to affect the vulnerable program, rather than hijacking the process completely, though it is possible. You
can, however, create detailed stack dumps, and thus the vulnerability can often serve to enhance knowledge of the
system. This knowledge can then be leveraged in other attacks - for example, if this vulnerability exists alongside a
buffer overflow, then the damage potential is significantly greater. In most situations, the presence of a stack canary
nullifies the risk of a buffer overflow attack. Using a format string vulnerability it is possible that an adversary
could tear values from the stack in seach of this canary, and duplicate it at the start of their NOP sled.
