\begin{figure}[h]
\centering
\begin{tabular}{|c|p{9.5cm}|l|}
\hline
{\bf Group Member} & {\bf Contribution Outline} & {\bf Contribution \%} \\
\hhline{|=|=|=|}
tj0824 & Hand-crafted buffer overflow shellcode, researched NX bits, overcame ASLR \& wrote technical details for attacks & A Noble 50\% \\
\hline
do6613 & Explored tools, automated advanced format string attacks \& stopped Tim from writing a small novel for report & A Saintly 50\% \\
\hline
\end{tabular}
\end{figure}

Together, we have exploited a pair of security issues in some very poorly written programs. Being experienced in various
security exploits, we decided to both attempt each task independently, resulting in subtly different attacks, which can
then be compared and contrasted for their various features. Whilst some steps in our attacks our similar or identical,
we did on occasion make effort to differ; using different format specifiers in the first lab, and finding alternative
methods to generate working shellcode in the second. Also of note is that we attacked both SEED virtual machines, for
the sake of completeness.

% For the second attack, we each converged on the same basic solution -- a standard NOP sled comprising a block of
% addresses, a large ``sled" made of around 450 NOP instructions, and a code payload at the very end. Execution was
% diverted to a point in the sled, allowing it to slide down into the payload. Tim's payload was written in x86 assembly
% which was then assembled with gcc and briefly analysed before being placed at the end of the sled. Drum instead used
% Metasploit to generate a payload for him, instructing the generator to launch /bin/sh and avoid null bytes.
