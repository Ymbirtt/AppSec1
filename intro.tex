\begin{figure}[h]
\centering
\begin{tabular}{|l|l|l|}
\hline
{\bf Group Member} & {\bf Contribution Outline} & {\bf Contribution \%} \\
\hline
tj0824 & Did some stuff & Probably 50\% \\
do6613 & Did some other stuff & Probably also 50\% \\
\hline
\end{tabular}
\end{figure}

Together, we have exploted a pair of security issues in some very poorly written programs. Being experienced in various
security exploits, we decided to both attempt each task independently, resulting in subtly different attacks, which can
then be compared and contrasted for their various features. Whilst some steps in our attacks our similar or identical,
we did on occasion make effort to differ, such as with our choice of format specifiers and our methods of
shellcode generation.

% For the second attack, we each converged on the same basic solution -- a standard NOP sled comprising a block of
% addresses, a large ``sled" made of around 450 NOP instructions, and a code payload at the very end. Execution was
% diverted to a point in the sled, allowing it to slide down into the payload. Tim's payload was written in x86 assembly
% which was then assembled with gcc and briefly analysed before being placed at the end of the sled. Drum instead used
% Metasploit to generate a payload for him, instructing the generator to launch /bin/sh and avoid null bytes.
